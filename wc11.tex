\documentclass[a4paper]{exam}

\usepackage{amsmath}
\usepackage{geometry}
\usepackage{hyperref}
\usepackage{pythonhighlight}

\printanswers

\title{Weekly Challenge 11: Edit Distance\\CS 412 Algorithms: Design and Analysis}
\author{@username}  % replace with your GitHub user names
\date{Habib University\\Spring 2022}

\qformat{{\large\bf \thequestion. \thequestiontitle}\hfill}
\boxedpoints

\begin{document}
\maketitle

\begin{questions}
  
\titledquestion{Edit Distance}

  Given a set of string operations, the \textit{edit distance} from string \texttt{x} to string \texttt{y} is the cost of the least expensive operation sequence that transforms \texttt{x} to \texttt{y}. It is described in detail in Problem 15-5 of CLRS and in Section 6.3 of Dasgupta et al. There are slight differences in the explanations and, more importantly, the allowed operations in each book. This problem follows the latter. You will be pleased to see that the \textit{edit distance} problem can be framed as a minor variation of the \textit{longest common subsequence} problem.

  The tasks below will ask you to output the filled table for the problem. For the table, follow the convention set in Figure 15.8 of CLRS. Note that the arrows in this figure are different from those used in Figures 6.4 and 6.5 in Dasgupta et al. 

  \textbf{TASKS}:
  \begin{parts}
  \part Write a function that takes two strings as parameters and outputs the filled table corresponding to the computation of the edit distance from the first to the second. The format of the table should be as described above. The file format is up to you--you may write it to an html file, to a \LaTeX file, or any other suitable format.
  \part Include below the rendered tables for some strings of your choice. Feel free to include any notable observations, e.g. is edit distance \href{https://en.wikipedia.org/wiki/Commutative_property}{commutative}?
  \part Share your tables in a comment on the \textit{Week 11 Challenge} post in the course group on Yammer.
  \end{parts}
 
  \begin{solution}
    % Enter your solution here.
  \end{solution}
\end{questions}
\end{document}

%%% Local Variables:
%%% mode: latex
%%% TeX-master: t
%%% End: